\documentclass[DM,lsstdraft,STR,toc]{lsstdoc}
\usepackage{geometry}
\usepackage{longtable,booktabs}
\usepackage{enumitem}
\usepackage{arydshln}

\input meta.tex

\providecommand{\tightlist}{
  \setlength{\itemsep}{0pt}\setlength{\parskip}{0pt}}

\begin{document}

\def\milestoneName{Spectrograph data acquisition}
\def\milestoneId{LDM-503-8}
\def\product{Data Management}

\setDocCompact{true}

\title[\milestoneId{}~Test Report]{\milestoneId{} (\milestoneName{})~Test Plan and Report}
\setDocRef{\lsstDocType-\lsstDocNum}
\setDocDate{\vcsdate}
\setDocUpstreamLocation{\url{https://github.com/lsst/lsst-texmf/examples}}
\author{ Michelle Butler }

\input history_and_info.tex


\setDocAbstract{
This is the test plan and report for \milestoneId{} (\milestoneName{}), an LSST level 2 milestone pertaining to the Data Management Subsystem.
}


\maketitle

\section{Introduction}
\label{sect:intro}


\subsection{Objectives}
\label{sect:objectives}

To acquire data from the spectrograph instrument along with proper
headers and transfer all data to NCSA for further viewing in LSP.~~



\subsection{System Overview}
\label{sect:systemoverview}

The spectragraph instrument is the first device to send real image data
to NCSA. ~ ~In the beginning, data is to be viewed before being stored
for record of the survey. ~Therefore data will be sent to NCSA for
further viewing through the LSST Science Platform (LSP). ~ The data can
be verified by a human scientist. ~If further processing or the data is
to be kept for further access, it can be ingested into the DBB along
with metadata and provedence data and then flagged for further
processing. ~ The version 1.0 of this test will have data written from
the spectragraph directly to a local filesystem on the ATS system. ~ A
script will be installed to copy all data from that image directory on
the ATS system to NCSA. ~The data will be placed into a directory at
NCSA, and a ButlerG2 record will be generated so that the LSP can access
the file. ~ After the data from the spectragraph will always want to be
kept, ~ version 2.0 of this test will be that all data from the
spectragraph will be placed on the L1 image uploader system. ~The L1
image uploader system will have the OODS service and the DBB service.
~The DBB service will transfer the data to NCSA and use the header files
to generate metadata/provedence data. ~The DBB will be responsible to
generate the ButlerG3 access files along with protect the data from
never loosing it, and ingest it into the permanent record of the survey.
~The L1 image uploader system will enable quick access to images that
were just taken and keep 30 days worth of cache for scientists to view
as needed. ~\\[2\baselineskip]

\subsection{Applicable Documents}\label{applicable-documents}

\citeds{LDM-294} Data Management Organization and Management\\
\citeds{LDM-503} DM Test Plan\\
\citeds{LDM-148} Data Management System Design


\subsection{Document Overview}
\label{sect:docoverview}

This document was generated from Jira, obtaining the relevant information from the 
\href{https://jira.lsstcorp.org/secure/Tests.jspa#/testPlan/LVV-P32}{LVV-P32}
~Jira Test Plan and related Test Cycles (
  \href{https://jira.lsstcorp.org/secure/Tests.jspa#/testCycle/LVV-C56}{LVV-C56}
).

Section \ref{sect:intro} provides an overview of the test campaign, the system under test (\product{}), the applicable documentation, and explains how this document is organized.
Section \ref{sect:configuration}  describes the configuration used for this test.
Section \ref{sect:personnel} describes the necessary roles and lists the individuals assigned to them.
%Section \ref{sect:plannedtestactivities} provides the list of planned test cycles and test cases, including all relevant information that fully describes the test campaign.

Section \ref{sect:overview} provides a summary of the test results, including an overview in Table \ref{table:summary}, an overall assessment statement and suggestions for possible improvements.
Section \ref{sect:detailedtestresults} provides detailed results for each step in each test case.

The current status of test plan LVV-P32 in Jira is Draft.

\subsection{References}
\label{sect:references}
\renewcommand{\refname}{}
\bibliography{lsst,refs,books,refs_ads}
\section{Test Configuration}
\label{sect:configuration}

\subsection{Data Collection}

  Observing is not required for this test campaign.

\subsection{Verification Environment}
\label{sect:hwconf}
  The verification will be that a file is placed on the ATS system by the
spectragraph, and then it's transferred and viewed by scientists in the
LSP. ~


  \subsection{Entry Criteria}
  Image data taken by the spectragraph.~~


  \subsection{Exit Criteria}
  Image data from spectragraph viewable by LSP at NCSA.~



\section{Personnel}
\label{sect:personnel}

The following personnel are involved in this test activity:

\begin{itemize}
\item Test Plan (LVV-P32) owner: Michelle Butler
\item Test Cycles:
\begin{itemize}
  \item LVV-C56 owner: 
    Undefined
  \begin{itemize}
    \item Test case LVV-T454 tester: 
  \end{itemize}
\end{itemize}
\item Additional Test Personnel involved: None
\end{itemize}

\newpage

\section{Overview of the Test Results}
\label{sect:overview}

\subsection{Summary}
\label{sect:summarytable}

\begin{longtable}{p{0.12\textwidth}p{0.2\textwidth}p{0.56\textwidth}p{0.12\textwidth}}
\toprule
  \multicolumn{3}{c}{ Test Cycle {\bf LVV-C56: Enable spectragraph data viewable by LSP.
 }} \\\hline
  {\bf \footnotesize test case} & {\bf \footnotesize status} & {\bf \footnotesize comment} & {\bf \footnotesize issues} \\\toprule
    \href{https://jira.lsstcorp.org/secure/Tests.jspa#/testCase/LVV-T454}{LVV-T454} 
    & Not Executed &  &
    \\\hline

\caption{Test Results Summary}
\label{table:summary}
\end{longtable}

\subsection{Overall Assessment}
\label{sect:overallassessment}

Not yet available.

\subsection{Recommended Improvements}
\label{sect:recommendations}

Not yet available.

\newpage
\section{Detailed Test Results}
\label{sect:detailedtestresults}


  \subsection{Test Cycle LVV-C56 }

Open test cycle {\it \href{https://jira.lsstcorp.org/secure/Tests.jspa#/testrun/LVV-C56}{Enable spectragraph data viewable by LSP.
}} in Jira.

  Enable spectragraph data viewable by LSP.
\\
  Status: Not Executed

  Image data needs to be created and available for transfer to NCSA. ~The
data will be transferred to NCSA and made available to scientists for
viewing and verification.~ ~~


  \subsubsection{Software Version/Baseline}
    Not provided.

  \subsubsection{Configuration}
    ATS system connected to spectragraph DAQ is required to have well formed
image data. ~Well formed means ``good'' image and correct headers.
~\\[2\baselineskip]


  \subsubsection{Test Cases in LVV-C56 Test Cycle}


    \paragraph{Test Case LVV-T454 }\mbox{}\\

Open  \href{https://jira.lsstcorp.org/secure/Tests.jspa#/testCase/LVV-T454}{\textit{ LVV-T454 } }
test case in Jira.

    \begin{itemize}
\tightlist
\item
  Acquire spectragraph image data, transfer that data to NCSA, ingest
  data into some Butler (G2 or G3 when available), and enable viewing of
  data on LSP. ~
\end{itemize}


    {\bf Preconditions}:\\
    Data must be well formed on Spectragraph data archiving system (ATS).
~Well-formed means that it has good headers and the file names are
unique. ~


    Execution status: {\bf Not Executed }

    Final comment:\\


    Detailed step results:

    \begin{longtable}{p{1cm}p{2cm}p{13cm}}
    \hline
    {Step} & \multicolumn{2}{c}{Description, Results and Status}\\ \hline
      1 & Description &

      \begin{minipage}[t]{13cm}{\footnotesize
      Have data on the ATS archiver system from the spectragraph.~

      \vspace{\dp0}
      } \end{minipage} \\
      \\ \cdashline{2-3}

      & Expected Result & 

      \begin{minipage}[t]{13cm}{\footnotesize
      Well formed files on the ATS system that need to be transferred to NCSA
for further analysis

      \vspace{\dp0}
      } \end{minipage} \\
      \\ \cdashline{2-3}

      & \begin{minipage}[t]{2cm}{Actual\\ Result}\end{minipage}   & 
      \begin{minipage}[t]{13cm}{\footnotesize
      
      \vspace{\dp0}
      } \end{minipage} \\
      \\ \cdashline{2-3}


      & Status          & Not Executed \\ \hline

      2 & Description &

      \begin{minipage}[t]{13cm}{\footnotesize
      human runs script to transfer data to NCSA through secure pipeline, or a
cronjob starts up data ``sync'' process.~~

      \vspace{\dp0}
      } \end{minipage} \\
      \\ \cdashline{2-3}

      & Expected Result & 

      \begin{minipage}[t]{13cm}{\footnotesize
      Data is transferred to NCSA, and is located in NCSA file
systems.\\[2\baselineskip]

      \vspace{\dp0}
      } \end{minipage} \\
      \\ \cdashline{2-3}

      & \begin{minipage}[t]{2cm}{Actual\\ Result}\end{minipage}   & 
      \begin{minipage}[t]{13cm}{\footnotesize
      
      \vspace{\dp0}
      } \end{minipage} \\
      \\ \cdashline{2-3}


      & Status          & Not Executed \\ \hline

      3 & Description &

      \begin{minipage}[t]{13cm}{\footnotesize
      All files transferred have a ButlerG2 (or G3 when ready) ingest
process.~

      \vspace{\dp0}
      } \end{minipage} \\
      \\ \cdashline{2-3}

      & Expected Result & 

      \begin{minipage}[t]{13cm}{\footnotesize
      files now can be accessed by Butler access methods\\[2\baselineskip]

      \vspace{\dp0}
      } \end{minipage} \\
      \\ \cdashline{2-3}

      & \begin{minipage}[t]{2cm}{Actual\\ Result}\end{minipage}   & 
      \begin{minipage}[t]{13cm}{\footnotesize
      
      \vspace{\dp0}
      } \end{minipage} \\
      \\ \cdashline{2-3}


      & Status          & Not Executed \\ \hline

      4 & Description &

      \begin{minipage}[t]{13cm}{\footnotesize
      LSP processes can now view spectragraph generate files~

      \vspace{\dp0}
      } \end{minipage} \\
      \\ \cdashline{2-3}

      & Expected Result & 

      \begin{minipage}[t]{13cm}{\footnotesize
      LSP jupyter notebooks can view spectragraph files.\\[2\baselineskip]

      \vspace{\dp0}
      } \end{minipage} \\
      \\ \cdashline{2-3}

      & \begin{minipage}[t]{2cm}{Actual\\ Result}\end{minipage}   & 
      \begin{minipage}[t]{13cm}{\footnotesize
      
      \vspace{\dp0}
      } \end{minipage} \\
      \\ \cdashline{2-3}


      & Status          & Not Executed \\ \hline

    \end{longtable}


\newpage
\appendix
%Make sure lsst-texmf/bin/generateAcronyms.py is in your path
\section{Acronyms used in this document}\label{sec:acronyms}
\addtocounter{table}{-1}
\begin{longtable}{p{0.145\textwidth}p{0.8\textwidth}}\hline
\textbf{Acronym} & \textbf{Description}  \\\hline

BDC &  Base Data Center \\\hline
DAQ & Data Acquisition System \\\hline
DBB & Data Back Bone \\\hline
DM & Data Management \\\hline
DTN & Data Transfer Node \\\hline
LDF & LSST Data Facility \\\hline
LDM & LSST Data Management (Document Handle) \\\hline
LSE & LSST Systems Engineering (Document Handle) \\\hline
LSP & LSST Science Platform \\\hline
LSST & Large Synoptic Survey Telescope \\\hline
NCSA & National Center for Supercomputing Applications \\\hline
OODS & Observatory Operations Data Service \\\hline
OS & Operating System \\\hline
\end{longtable}


\end{document}
