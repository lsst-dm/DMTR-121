\documentclass[DM,lsstdraft,STR,toc]{lsstdoc}
\usepackage{geometry}
\usepackage{longtable,booktabs}
\usepackage{enumitem}
\usepackage{arydshln}

\input meta.tex

\providecommand{\tightlist}{
  \setlength{\itemsep}{0pt}\setlength{\parskip}{0pt}}

\begin{document}

\def\milestoneName{Spectrograph Data Acquisition}
\def\milestoneId{LDM-503-8}
\def\product{Data Management}

\setDocCompact{true}

\title{ LDM-503-8 Spectrograph Data Acquisition Test Plan and Report}
\setDocRef{\lsstDocType-\lsstDocNum}
\date{\vcsdate}
\setDocUpstreamLocation{\url{https://github.com/lsst/lsst-texmf/examples}}
\author{ Michelle Butler }

\input history_and_info.tex


\setDocAbstract{
This is the test plan and report for LDM-503-8 (Spectrograph Data Acquisition), an LSST level 2 milestone pertaining to the Data Management Subsystem.
}


\maketitle

\section{Introduction}
\label{sect:intro}


\subsection{Objectives}
\label{sect:objectives}

To acquire data from the spectrograph instrument along with proper
headers and transfer all data to NCSA for further viewing in LSP. ~



\subsection{System Overview}
\label{sect:systemoverview}

The spectrograph instrument is the first device to have real data for
viewing and evaluation. ~ Early spectrograph data will undergo human
inspection to determine whether it is suitable for further processing;
if it is not, it will be deleted. ~ Therefore the image data will be
sent to NCSA for further viewing through the LSST Science Platform
(LSP). ~ ~If further processing is desired, the image will be ingested
into the DBB along with its metadata/provenance for further processing.
~ Version 1.0 of this test will have data written from the spectrograph
directly to a local disk on the ATS storage server which is co-located
and directly connected to the spectrograph instrument. ~ Data will be
copied from that local disk on the ATS storage server to NCSA and a
ButlerG2 (or G3) ingest process completed so that the LSP can access the
image data. ~ Version 2.0 of this test, when the spectrograph is on the
summit, will be that all data from the spectrograph will be written to
the L1 handoff system, which will be at the BDC (base data center). ~The
L1 handoff system will have the OODS(Observatory Operations Data
Service) service and the DBB(Data BackBone) service in place. ~The DBB
service will transfer the data to NCSA and be responsible for generating
metadata/provenance data from image headers. ~The DBB will be
responsible to ingest the image file into the correct ButlerG2 (or G3)
repository, and DBB repository. ~The L1 handoff system through the OODS
will enable quick access to images that were just created and keep 30
days worth of cached images for scientists to view as needed via the LSP
instance on the summit. ~\\[2\baselineskip]

\subsection{Applicable Documents}\label{applicable-documents}

\citeds{LDM-294} Data Management Organization and Management\\
\citeds{LDM-503} DM Test Plan\\
\citeds{LDM-148} Data Management System Design\\
\citeds{LSE-400} Header Service~


\subsection{Document Overview}
\label{sect:docoverview}

This document was generated from Jira, obtaining the relevant information from the 
\href{https://jira.lsstcorp.org/secure/Tests.jspa#/testPlan/LVV-P32}{LVV-P32}
~Jira Test Plan and related Test Cycles (
  \href{https://jira.lsstcorp.org/secure/Tests.jspa#/testCycle/LVV-C56}{LVV-C56}
).

Section \ref{sect:intro} provides an overview of the test campaign, the system under test (\product{}), the applicable documentation, and explains how this document is organized.
Section \ref{sect:configuration}  describes the configuration used for this test.
Section \ref{sect:personnel} describes the necessary roles and lists the individuals assigned to them.
%Section \ref{sect:plannedtestactivities} provides the list of planned test cycles and test cases, including all relevant information that fully describes the test campaign.

Section \ref{sect:overview} provides a summary of the test results, including an overview in Table \ref{table:summary}, an overall assessment statement and suggestions for possible improvements.
Section \ref{sect:detailedtestresults} provides detailed results for each step in each test case.

The current status of test plan LVV-P32 in Jira is \textbf{ Approved }.

\subsection{References}
\label{sect:references}
\renewcommand{\refname}{}
\bibliography{lsst,refs,books,refs_ads}
\section{Test Configuration}
\label{sect:configuration}

\subsection{Data Collection}

  Observing is not required for this test campaign.

\subsection{Verification Environment}
\label{sect:hwconf}
  Verification environment will be the ATS storage server attached to the
spectrograph. ~ The OS is Centos 6.* running latest security patches. ~A
Cron script is used to move to images ~on ATS storage server to NCSA. ~
IP tables and port verifications are used for data transfer to NCSA.
~The receiving end at NCSA file systems is a file system ~available with
the DTN protocol configured and set up. ~ NCSA verification scripts
including ButlerG* ingests are run. ~\\[3\baselineskip]


  \subsection{Entry Criteria}
  Image data taken by the spectrograph. ~


  \subsection{Exit Criteria}
  Image data from spectrograph viewable by LSP at NCSA.~



\newpage
\section{Personnel}
\label{sect:personnel}

The following personnel are involved in this test activity:

\begin{itemize}
\item Test Plan (LVV-P32) owner: Michelle Butler
\item Test Cycles:
\begin{itemize}
  \item LVV-C56 owner: 
    Michelle Gower
  \begin{itemize}
    \item Test case \href{https://jira.lsstcorp.org/secure/Tests.jspa#/testCase/LVV-T454}{LVV-T454} tester: Michelle Butler
  \end{itemize}
\end{itemize}
\item Additional Test Personnel involved:
  \begin{itemize}
    \item Test case \href{https://jira.lsstcorp.org/secure/Tests.jspa#/testCase/LVV-T454}{LVV-T454}: Michelle Gower, Htut-Khine, SRP

  \end{itemize}
\end{itemize}

\newpage

\section{Overview of the Test Results}
\label{sect:overview}

\subsection{Summary}
\label{sect:summarytable}

\begin{longtable}{p{0.12\textwidth}p{0.2\textwidth}p{0.56\textwidth}p{0.12\textwidth}}
\toprule

  \multicolumn{3}{c}{ Test Cycle {\bf LVV-C56: LDM-503-8 Enable spectrograph data viewable by LSP.
 }} \\\hline

  {\bf \footnotesize test case} & {\bf \footnotesize status} & {\bf \footnotesize comment} & {\bf \footnotesize issues} \\\toprule

    \href{https://jira.lsstcorp.org/secure/Tests.jspa#/testCase/LVV-T454}{LVV-T454}
    & Pass & executed with no problems.~
 &
    \\\hline

\caption{Test Results Summary}
\label{table:summary}
\end{longtable}

\subsection{Overall Assessment}
\label{sect:overallassessment}

The test worked well, and data has been flowing from the AT-archiver in
Tucson to the LDF ever since. ~ There have been some hiccups in the
ingestion process due to the stack changing, but have been fixed with
releases of the stack in following weeks. ~ ~The data flow right now is
done by Rsync, but will be switched out to something immediately
transmit without cron when the image file is created, but that won't be
until Feb 2020.~~


\subsection{Recommended Improvements}
\label{sect:recommendations}

There is no need for the image to be examined before it's ingested into
the LDF. ~ There are too many images at this time to evaluate them at
that rate, and many want to see the image quickly. ~Having the image
examined by the science staff before i's ingested could be a long delay.
~ ~It would be better that the science staff examine the images after
they have been ingested and on the LDF.~~


\newpage
\section{Detailed Test Results}
\label{sect:detailedtestresults}


  \subsection{Test Cycle LVV-C56 }

Open test cycle {\it \href{https://jira.lsstcorp.org/secure/Tests.jspa#/testrun/LVV-C56}{LDM-503-8 Enable spectrograph data viewable by LSP.
}} in Jira.

  LDM-503-8 Enable spectrograph data viewable by LSP.
\\
  Status: Done

  Image data needs to be created and available for transfer to NCSA. ~The
data will be transferred to NCSA and made available to scientists for
viewing and verification.~ ~~


  \subsubsection{Software Version/Baseline}
    Not provided.

  \subsubsection{Configuration}
    ATS system connected to spectrograph DAQ is required to have well formed
image data. ~Well formed means ``good'' image and correct headers
(\citeds{LSE-400}). ~\\[2\baselineskip]


  \subsubsection{Test Cases in LVV-C56 Test Cycle}


    \paragraph{Test Case LVV-T454 - LDM-503-8 Enable LSP viewing of spectrograph data.
 }\mbox{}\\

Open  \href{https://jira.lsstcorp.org/secure/Tests.jspa#/testCase/LVV-T454}{\textit{ LVV-T454 } }
test case in Jira.

    \begin{itemize}
\tightlist
\item
  Acquire spectrograph image data, transfer that data to NCSA, ingest
  data into a Butler (G2 or G3 when available), and enable viewing of
  data on LSP. ~
\end{itemize}


    \textbf{ Preconditions}:\\
    Data must be well formed on Spectrograph data archiving system (ATS).
~Well-formed means ``good image'' and correct headers. (\citeds{LSE-400}) ~


    Execution status: {\bf Pass }

    Final comment:\\executed with no problems.~



    Detailed step results:

    \begin{longtable}{p{1cm}p{2cm}p{13cm}}
    \hline
    {Step} & \multicolumn{2}{c}{Description, Results and Status}\\ \hline
      1 & Description &

      \begin{minipage}[t]{13cm}{\footnotesize
      Have data on the ATS archiver system from the spectrograph.~

      \vspace{\dp0}
      } \end{minipage} \\
      \\ \cdashline{2-3}


      & Expected Result &

      \begin{minipage}[t]{13cm}{\footnotesize
      Well formed files on the ATS system that need to be transferred to NCSA
for further analysis

      \vspace{\dp0}
      } \end{minipage} \\
      \\ \cdashline{2-3}

      & \begin{minipage}[t]{2cm}{Actual\\ Result}\end{minipage}   & 
      \begin{minipage}[t]{13cm}{\footnotesize
      
      \vspace{\dp0}
      } \end{minipage} \\
      \\ \cdashline{2-3}


      & Status          & Pass \\ \hline

      2 & Description &

      \begin{minipage}[t]{13cm}{\footnotesize
      A first few iterations is the human runs script to transfer data to NCSA
through secure pipeline. ~after the process is unchanging/solid, a
cronjob starts up data ``sync'' process. ~

      \vspace{\dp0}
      } \end{minipage} \\
      \\ \cdashline{2-3}


      & Expected Result &

      \begin{minipage}[t]{13cm}{\footnotesize
      Data is transferred to NCSA, and is located in NCSA file
systems.\\[2\baselineskip]

      \vspace{\dp0}
      } \end{minipage} \\
      \\ \cdashline{2-3}

      & \begin{minipage}[t]{2cm}{Actual\\ Result}\end{minipage}   & 
      \begin{minipage}[t]{13cm}{\footnotesize
      
      \vspace{\dp0}
      } \end{minipage} \\
      \\ \cdashline{2-3}


      & Status          & Pass \\ \hline

      3 & Description &

      \begin{minipage}[t]{13cm}{\footnotesize
      All files transferred have a ButlerG2 (or G3 when ready) ingest
process.~

      \vspace{\dp0}
      } \end{minipage} \\
      \\ \cdashline{2-3}


      & Expected Result &

      \begin{minipage}[t]{13cm}{\footnotesize
      files now can be accessed by Butler access methods\\[2\baselineskip]

      \vspace{\dp0}
      } \end{minipage} \\
      \\ \cdashline{2-3}

      & \begin{minipage}[t]{2cm}{Actual\\ Result}\end{minipage}   & 
      \begin{minipage}[t]{13cm}{\footnotesize
      
      \vspace{\dp0}
      } \end{minipage} \\
      \\ \cdashline{2-3}


      & Status          & Pass \\ \hline

      4 & Description &

      \begin{minipage}[t]{13cm}{\footnotesize
      LSP processes can now view spectrograph generate files~

      \vspace{\dp0}
      } \end{minipage} \\
      \\ \cdashline{2-3}


      & Expected Result &

      \begin{minipage}[t]{13cm}{\footnotesize
      LSP jupyter notebooks can view spectrograph files.\\[2\baselineskip]

      \vspace{\dp0}
      } \end{minipage} \\
      \\ \cdashline{2-3}

      & \begin{minipage}[t]{2cm}{Actual\\ Result}\end{minipage}   & 
      \begin{minipage}[t]{13cm}{\footnotesize
      
      \vspace{\dp0}
      } \end{minipage} \\
      \\ \cdashline{2-3}


      & Status          & Pass \\ \hline

    \end{longtable}


\newpage
\appendix
%Make sure lsst-texmf/bin/generateAcronyms.py is in your path
\section{Acronyms used in this document}\label{sec:acronyms}
\addtocounter{table}{-1}
\begin{longtable}{p{0.145\textwidth}p{0.8\textwidth}}\hline
\textbf{Acronym} & \textbf{Description}  \\\hline

BDC &  Base Data Center \\\hline
DAQ & Data Acquisition System \\\hline
DBB & Data Back Bone \\\hline
DM & Data Management \\\hline
DTN & Data Transfer Node \\\hline
LDF & LSST Data Facility \\\hline
LDM & LSST Data Management (Document Handle) \\\hline
LSE & LSST Systems Engineering (Document Handle) \\\hline
LSP & LSST Science Platform \\\hline
LSST & Large Synoptic Survey Telescope \\\hline
NCSA & National Center for Supercomputing Applications \\\hline
OODS & Observatory Operations Data Service \\\hline
OS & Operating System \\\hline
\end{longtable}


\end{document}
