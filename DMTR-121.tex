\documentclass[DM,lsstdraft,STR,toc]{lsstdoc}
\usepackage{geometry}
\usepackage{longtable,booktabs}
\usepackage{enumitem}
\usepackage{arydshln}

\input meta.tex

\providecommand{\tightlist}{
  \setlength{\itemsep}{0pt}\setlength{\parskip}{0pt}}

\setcounter{tocdepth}{4}

\begin{document}

\def\milestoneName{Spectrograph Data Acquisition}
\def\milestoneId{LDM-503-8}
\def\product{Data Management}

\setDocCompact{true}

\title{ LDM-503-8 Spectrograph Data Acquisition Test Plan and Report}
\setDocRef{\lsstDocType-\lsstDocNum}
\date{\vcsdate}
\setDocUpstreamLocation{\url{https://github.com/lsst/lsst-texmf/examples}}
\author{ Michelle Butler }

\input history_and_info.tex


\setDocAbstract{
This is the test plan and report for LDM-503-8 (Spectrograph Data Acquisition),
an LSST level 2 milestone pertaining to the Data Management Subsystem.
}


\maketitle

\section{Introduction}
\label{sect:intro}


\subsection{Objectives}
\label{sect:objectives}

To acquire data from the spectrograph instrument along with proper
headers and transfer all data to NCSA for further viewing in LSP. ~



\subsection{System Overview}
\label{sect:systemoverview}

The spectrograph instrument is the first device to have real data for
viewing and evaluation. ~ Early spectrograph data will undergo human
inspection to determine whether it is suitable for further processing;
if it is not, it will be deleted. ~ Therefore the image data will be
sent to NCSA for further viewing through the LSST Science Platform
(LSP). ~ ~If further processing is desired, the image will be ingested
into the DBB along with its metadata/provenance for further processing.
~ Version 1.0 of this test will have data written from the spectrograph
directly to a local disk on the ATS storage server which is co-located
and directly connected to the spectrograph instrument. ~ Data will be
copied from that local disk on the ATS storage server to NCSA and a
ButlerG2 (or G3) ingest process completed so that the LSP can access the
image data. This test was run with Butler/G2.~ ~Version 2.0 of this
test, when the spectrograph is on the summit, will be that all data from
the spectrograph will be written to the L1 handoff system, which will be
at the BDC (base data center). ~The L1 handoff system will have the
OODS(Observatory Operations Data Service) service and the DBB(Data
BackBone) service in place. ~The DBB service will transfer the data to
NCSA and be responsible for generating metadata/provenance data from
image headers. ~The DBB will be responsible to ingest the image file
into the correct ButlerG2 (or G3) repository, and DBB repository. ~The
L1 handoff system through the OODS will enable quick access to images
that were just created and keep 30 days worth of cached images for
scientists to view as needed via the LSP instance on the summit.
~\\[2\baselineskip]

\subsection{Applicable Documents}\label{applicable-documents}

\citeds{LDM-294} Data Management Organization and Management\\
\citeds{LDM-503} DM Test Plan\\
\citeds{LDM-148} Data Management System Design\\
\citeds{LSE-400} Header Service~


\subsection{Document Overview}
\label{sect:docoverview}

This document was generated from Jira, obtaining the relevant information from the 
\href{https://jira.lsstcorp.org/secure/Tests.jspa#/testPlan/LVV-P32}{LVV-P32}
~Jira Test Plan and related Test Cycles (
  \href{https://jira.lsstcorp.org/secure/Tests.jspa#/testCycle/LVV-C56}{LVV-C56}
).

Section \ref{sect:intro} provides an overview of the test campaign, the system under test (\product{}),
the applicable documentation, and explains how this document is organized.
Section \ref{sect:configuration}  describes the configuration used for this test.
Section \ref{sect:personnel} describes the necessary roles and lists the individuals assigned to them.
%Section \ref{sect:plannedtestactivities} provides the list of planned test cycles and test cases,
including all relevant information that fully describes the test campaign.

Section \ref{sect:overview} provides a summary of the test results, including an overview in Table \ref{table:summary},
an overall assessment statement and suggestions for possible improvements.
Section \ref{sect:detailedtestresults} provides detailed results for each step in each test case.

The current status of test plan LVV-P32 in Jira is \textbf{ Completed }.

\subsection{References}
\label{sect:references}
\renewcommand{\refname}{}
\bibliography{lsst,refs,books,refs_ads}
\section{Test Configuration}
\label{sect:configuration}

\subsection{Data Collection}

  Observing is not required for this test campaign.

\subsection{Verification Environment}
\label{sect:hwconf}
  Verification environment will be the ATS storage server attached to the
spectrograph. ~ The OS is Centos 6.* running latest security patches. ~A
Cron script is used to move to images ~on ATS storage server to NCSA. ~
IP tables and port verifications are used for data transfer to NCSA.
~The receiving end at NCSA file systems is a file system ~available with
the DTN protocol configured and set up. ~ NCSA verification scripts
including ButlerG* ingests are run. ~\\[3\baselineskip]


  \subsection{Entry Criteria}
  Image data taken by the spectrograph. ~


  \subsection{Exit Criteria}
  Image data from spectrograph viewable by LSP at NCSA.~



\newpage
\section{Personnel}
\label{sect:personnel}

The personnel involved in the test campaign are shown in the following table.

\begin{longtable}{p{3cm}p{3cm}p{3cm}p{6cm}}
\hline
\multicolumn{2}{r}{Test Plan (LVV-P32) owner:} &
\multicolumn{2}{l}{\textbf{ Michelle Butler } }\\\hline
\multicolumn{2}{r}{ LVV-C56 owner:} &
\multicolumn{2}{l}{\textbf{
    Michelle Gower
}
} \\\hline
\textbf{Test Case} & \textbf{Assigned to} & \textbf{Executed by} & \textbf{Additional Test Personnel} \\ \hline
\href{https://jira.lsstcorp.org/secure/Tests.jspa#/testCase/LVV-T454}{LVV-T454}
& {\small Michelle Butler } & {\small Michelle Butler } &
\begin{minipage}[]{6cm}
\smallskip
{\small Michelle Gower, Htut-Khine, SRP
 }
\medskip
\end{minipage}
\\ \hline
\end{longtable}

\newpage

\section{Test Campaign Overview}
\label{sect:overview}

\subsection{Summary}
\label{sect:summarytable}

\begin{longtable}{p{2cm}p{2.5cm}p{9cm}p{2.5cm}}
\toprule
\multicolumn{3}{l}{ Test Plan {\bf LVV-P32: LDM-503-8 Spectrograph Data Acquisition
 }} & Completed \\\hline

  \multicolumn{3}{l}{ Test Cycle {\bf LVV-C56: LDM-503-8 Enable spectrograph data viewable by LSP.
 }} & Done \\\hline

  {\bf \footnotesize test case} & {\bf \footnotesize status} & {\bf \footnotesize comment} & {\bf \footnotesize issues} \\\toprule

\href{https://jira.lsstcorp.org/secure/Tests.jspa#/testCase/LVV-T454}{LVV-T454}
    & Pass &
    \begin{minipage}[]{9cm}
    \smallskip
    executed with no problems.~

    \medskip
    \end{minipage}
    &
    \\\hline
\caption{Test Campaign Summary}
\label{table:summary}
\end{longtable}

\subsection{Overall Assessment}
\label{sect:overallassessment}

The test worked well, and data has been flowing from the AT-archiver in
Tucson to the LDF ever since. ~ There have been some hiccups in the
ingestion process due to the stack changing, but have been fixed with
releases of the stack in following weeks. ~This is how data is brought
to NCSA from the test stand in Tucson on a daily basis. ~ The specific
time it was run for this test was 8/20 and the stack version is
w\_2019\_33. ~All the files from the Tucson test stand are ingested and
placed in this path:
/lsstdata/offline/teststand/auxTel/L1Archiver/storage/ ~ ~The data flow
right now is done by Rsync, but will be switched out to something that
will immediately transmit a file without cron when the image file is
created, but that won't be until Feb 2020. ~ ~All the files have all
been re-ingested this week with stack W\_2019\_50: ~creating new sqllite
repos for all images. ~(12/17/2019)~


\subsection{Recommended Improvements}
\label{sect:recommendations}

There is no need for the image to be examined before it's ingested into
the LDF. ~ There are too many images at this time to evaluate them at
that rate, and many want to see the image quickly. ~Having the image
examined by the science staff before it's ingested could be a long
delay. ~ ~It would be better that the science staff examine the images
after they have been ingested and on the LDF.~~


\newpage
\section{Detailed Test Results}
\label{sect:detailedtestresults}

\subsection{Test Cycle LVV-C56 }

Open test cycle {\it \href{https://jira.lsstcorp.org/secure/Tests.jspa#/testrun/LVV-C56}{LDM-503-8 Enable spectrograph data viewable by LSP.
}} in Jira.

LDM-503-8 Enable spectrograph data viewable by LSP.
\\
Status: Done

Image data needs to be created and available for transfer to NCSA. ~The
data will be transferred to NCSA and made available to scientists for
viewing and verification.~ ~~


\subsubsection{Software Version/Baseline}
The transfer of the file does not use versions, but the ingest of the
files into Butler/G2 requires the weekly stack and the LSSTobj
environment. ~ ~All files from the Tucson test stand are transferred to
NCSA in this fashion and then injested. ~Files are ingested with the
``current'' weekly when the file is transferred at the time. ~ For
example ~image file: AT\_O\_20190820\_000001-ats-wfs\_ccd.fits was
ingested with W\_2019\_33, and can be found in the file system on the
LSP:
/lsstdata/offline/teststand/auxTel/L1Archiver/gen2repo/raw/2019-08-20/2019082000001-det000.fits.
~As an update to this test; ~All files as of (12/17/2019) have been
reingested with w\_2019\_50. ~ This fixed some problems with some of the
DB fields in the sqlite repos.~ ~


\subsubsection{Configuration}
ATS system connected to spectrograph DAQ is required to have well formed
image data. ~Well formed means ``good'' image and correct headers
(\citeds{LSE-400}). ~\\[2\baselineskip]


\subsubsection{Test Cases in LVV-C56 Test Cycle}

\paragraph{Test Case LVV-T454 - LDM-503-8 Enable LSP viewing of spectrograph data.
 }\mbox{}\\

Open  \href{https://jira.lsstcorp.org/secure/Tests.jspa#/testCase/LVV-T454}{\textit{ LVV-T454 } }
test case in Jira.

\begin{itemize}
\tightlist
\item
  Acquire spectrograph image data, transfer that data to NCSA, ingest
  data into a Butler (G2 or G3 when available), and enable viewing of
  data on LSP. ~
\end{itemize}


\textbf{ Preconditions}:\\
Data must be well formed on Spectrograph data archiving system (ATS).
~Well-formed means ``good image'' and correct headers. (\citeds{LSE-400}) ~


Execution status: {\bf Pass }

Final comment:\\executed with no problems.~



Detailed steps results:

\begin{longtable}{p{1cm}p{15cm}}
\hline
{Step} & Step Details\\ \hline
1 & Description \\
 & \begin{minipage}[t]{15cm}
{\footnotesize
Have data on the ATS archiver system from the spectrograph.~

\medskip }
\end{minipage}
\\ \cdashline{2-2}


 & Expected Result \\
 & \begin{minipage}[t]{15cm}{\footnotesize
Well formed files on the ATS system that need to be transferred to NCSA
for further analysis

\medskip }
\end{minipage} \\ \cdashline{2-2}

 & Actual Result \\
 & \begin{minipage}[t]{15cm}{\footnotesize

\medskip }
\end{minipage} \\ \cdashline{2-2}

 & Status: \textbf{ Pass } \\ \hline

2 & Description \\
 & \begin{minipage}[t]{15cm}
{\footnotesize
A first few iterations is the human runs script to transfer data to NCSA
through secure pipeline. ~after the process is unchanging/solid, a
cronjob starts up data ``sync'' process. ~

\medskip }
\end{minipage}
\\ \cdashline{2-2}


 & Expected Result \\
 & \begin{minipage}[t]{15cm}{\footnotesize
Data is transferred to NCSA, and is located in NCSA file
systems.\\[2\baselineskip]

\medskip }
\end{minipage} \\ \cdashline{2-2}

 & Actual Result \\
 & \begin{minipage}[t]{15cm}{\footnotesize

\medskip }
\end{minipage} \\ \cdashline{2-2}

 & Status: \textbf{ Pass } \\ \hline

3 & Description \\
 & \begin{minipage}[t]{15cm}
{\footnotesize
All files transferred have a ButlerG2 (or G3 when ready) ingest
process.~

\medskip }
\end{minipage}
\\ \cdashline{2-2}


 & Expected Result \\
 & \begin{minipage}[t]{15cm}{\footnotesize
files now can be accessed by Butler access methods\\[2\baselineskip]

\medskip }
\end{minipage} \\ \cdashline{2-2}

 & Actual Result \\
 & \begin{minipage}[t]{15cm}{\footnotesize

\medskip }
\end{minipage} \\ \cdashline{2-2}

 & Status: \textbf{ Pass } \\ \hline

4 & Description \\
 & \begin{minipage}[t]{15cm}
{\footnotesize
LSP processes can now view spectrograph generate files~

\medskip }
\end{minipage}
\\ \cdashline{2-2}


 & Expected Result \\
 & \begin{minipage}[t]{15cm}{\footnotesize
LSP jupyter notebooks can view spectrograph files.\\[2\baselineskip]

\medskip }
\end{minipage} \\ \cdashline{2-2}

 & Actual Result \\
 & \begin{minipage}[t]{15cm}{\footnotesize
from the terminal window on the LSP:\\
\$ ls\\
AT\_O\_20190820\_000001-ats-wfs\_ccd.fits
AT\_O\_20190820\_000005-ats-wfs\_ccd.fits
AT\_O\_20190820\_000009-ats-wfs\_ccd.fits\\
AT\_O\_20190820\_000002-ats-wfs\_ccd.fits
AT\_O\_20190820\_000006-ats-wfs\_ccd.fits
AT\_O\_20190820\_000010-ats-wfs\_ccd.fits\\
AT\_O\_20190820\_000003-ats-wfs\_ccd.fits
AT\_O\_20190820\_000007-ats-wfs\_ccd.fits
AT\_O\_20190820\_000011-ats-wfs\_ccd.fits\\
AT\_O\_20190820\_000004-ats-wfs\_ccd.fits
AT\_O\_20190820\_000008-ats-wfs\_ccd.fits
AT\_O\_20190820\_000012-ats-wfs\_ccd.fits\\
{[}mbutler@lsst-dev01 2019-08-20{]}\$ pwd\\
/lsstdata/offline/teststand/auxTel/L1Archiver/storage/2019-08-20\\[2\baselineskip]This
is just the listing of the files transferred and can be used as a user
wishes.~ ~\\[2\baselineskip]

\medskip }
\end{minipage} \\ \cdashline{2-2}

 & Status: \textbf{ Pass } \\ \hline

\end{longtable}


\newpage
\appendix
%Make sure lsst-texmf/bin/generateAcronyms.py is in your path
\section{Acronyms used in this document}\label{sec:acronyms}
\addtocounter{table}{-1}
\begin{longtable}{p{0.145\textwidth}p{0.8\textwidth}}\hline
\textbf{Acronym} & \textbf{Description}  \\\hline

BDC &  Base Data Center \\\hline
DAQ & Data Acquisition System \\\hline
DBB & Data Back Bone \\\hline
DM & Data Management \\\hline
DTN & Data Transfer Node \\\hline
LDF & LSST Data Facility \\\hline
LDM & LSST Data Management (Document Handle) \\\hline
LSE & LSST Systems Engineering (Document Handle) \\\hline
LSP & LSST Science Platform \\\hline
LSST & Large Synoptic Survey Telescope \\\hline
NCSA & National Center for Supercomputing Applications \\\hline
OODS & Observatory Operations Data Service \\\hline
OS & Operating System \\\hline
\end{longtable}


\end{document}
